DBpedia is an example of linked data, where structured data has been extracted
from Wikipedia. This data may be semantically queried, and as such contains
Wikipedia entries with metadata\cite{bizer2009dbpedia}.

DBpedia is designed for semantic querying of data, not presenting data. If
permitted to beg the question: data becomes more useful if used. So let's use
it.

Some of the entries have time-related metadata. This data might be more
absorbing if presented as an interactive visualisation.

We may scrape all of DBpedia, and record all of the entries with time metadata 
in tempuhs. This data may in turn be visualised using mytimelines.org.
