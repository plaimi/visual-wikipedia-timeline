


\begin{figure}[H]
  \centering
  \begin{scale}{0.3}
    \input{fig/overload.pdf_t}
  \end{scale}
  \caption{A timeline without filtering results in an information overload.}
  \label{fig:overload}
\end{figure}

Figure~\ref{fig:overload} demonstrates how a timeline of all of DBpedia might 
look like. The programmer art is admittedly a slight degradation of the user 
interface presented in January.

\begin{figure}[H]
  \centering
  \begin{scale}{0.3}
    \input{fig/filtered.pdf_t}
  \end{scale}
  \caption{A filtered timeline.}
  \label{fig:filtered}
\end{figure}

Using all of DBpedia without the option to filter on categories would be 
cumbersome. Filtering was not present functionality in the January 
presentation, so we have taken the liberty of augmenting the design in 
Figure~\ref{fig:filtered}. This allows fine-grained control of what data is 
presently presented.

\begin{figure}[H]
  \centering
  \begin{scale}{0.5}
    \input{fig/filters.pdf_t}
  \end{scale}
  \caption{Adding a new filter.}
  \label{fig:filters}
\end{figure}

The massive amount of data comes with a massive amount of categories for 
cataloguing purposes. A filter of filters is thus necessitated, as illustrated 
in Figure~\ref{fig:filters}.

\begin{figure}[H]
  \centering
  \begin{scale}{0.3}
    \input{fig/focussed.pdf_t}
  \end{scale}
  \caption{A focussed timespan.}
  \label{fig:focussed}
\end{figure}

Finally, Figure~\ref{fig:focussed} shows an expanded timespan, wherein the 
preamble of the relevant Wikipedia article is displayed along with the 
relevant article image.
