A more mature mytimelines.org will be able to realise our idea. Our idea is 
elegantly simple and simply elegant: make a stupendous timeline of nearly all 
of Wikipedia.

The timeline will consist of merely ``nearly'' all articles, because not every 
article has time metadata. The ones that do may however be scraped for such 
metadata via DBpedia. We have already done that.

\begin{figure}[H]
  \centering
  \begin{scale}{0.3}
    \input{fig/overload.pdf_t}
  \end{scale}
  \caption{A timeline without filtering results in an information overload.}
  \label{fig:overload}
\end{figure}

The reason we want one colossal timeline is so that the users themselves can 
judge what information is relevant through filtering the timespans contained 
within the timeline.

Figure~\ref{fig:overload} demonstrates how a timeline of all of DBpedia might 
look like. The programmer art is admittedly a slight degradation of the user 
interface presented in January. This figure demonstrates the problem of having 
Wikipedia as solely a visual timeline -- too much data! It's positively 
bewildering. Not cool.

\begin{figure}[H]
  \centering
  \begin{scale}{0.3}
    \input{fig/filtered.pdf_t}
  \end{scale}
  \caption{A filtered timeline.}
  \label{fig:filtered}
\end{figure}

Using all of DBpedia without the option to filter on categories would be
cumbersome at best. While filtering was not present functionality in the
January presentation, we have taken the liberty of augmenting the design in
Figure~\ref{fig:filtered}. This allows fine-grained control of what data is
presently presented. Now we have an interactive visual timeline -- very cool.

\begin{figure}[H]
  \centering
  \begin{scale}{0.5}
    \input{fig/filters.pdf_t}
  \end{scale}
  \caption{Adding a new filter.}
  \label{fig:filters}
\end{figure}

Additionally, it might be interesting to see articles that are related but not 
necessary in the same categories. Consequently there should be a way to make 
sets of filters, and express conjunctions and disjunctions of these. We have 
illustrated this in Figure~\ref{fig:filteropss}, but concede that a 
professional user interface designer might be capable of designing expressing 
this more pleasantly.

\begin{figure}[H]
  \centering
  \begin{scale}{0.5}
    \input{fig/filterops.pdf_t}
  \end{scale}
  \caption{Using filter operations to get both German physicists and Swiss Chemists from the 1900s.}
  \label{fig:filteropss}
\end{figure}

The massive amount of data comes with a massive amount of categories for 
cataloguing purposes. A filter of filters is thus necessitated, as illustrated 
in Figure~\ref{fig:filters}.

\begin{figure}[H]
  \centering
  \begin{scale}{0.3}
    \input{fig/focussed.pdf_t}
  \end{scale}
  \caption{A focussed timespan.}
  \label{fig:focussed}
\end{figure}

Finally, Figure~\ref{fig:focussed} shows an expanded timespan, wherein the 
preamble of the relevant Wikipedia article is displayed along with the 
relevant article image.
