We may scrape all of DBpedia, and record all of the entries with time metadata 
in tempuhs. This data may in turn be visualised using mytimelines.org.

In fact, DBpedia is already scraped and put into tempuhs, with metadata 
describing and cataloguing the entries. The cataloguing is important for being
able to filter the immense amounts of data.

Given a perfect universe with no computational bottlenecks, tempuhs is already 
fit for fight. Notwithstanding, due to the quantity of entries contained 
within DBpedia, a pragmatic timeline will require some improvements in tempuhs 
in order to serve it quickly enough. It needs to be partially served. It 
should furthermore in all likelihood be cached in a clever way. For this it 
might be fruitful to introduce the concept of read-only timespans to tempuhs.

As for mytimelines.org, it really is not up to par to handle this amount of 
data, or anything anywhere near this right now. timeline.js simply won't cut 
it. It is optimised for 20-30 entries\cite{timelinejsfaq} -- DBpedia will be 
several millions of entries. mytimelines.org in its current state will not be 
able to present the required amount of data. Perchance it will not even be 
able to *receive* the required amount of data.

The envisioned mytimelines.org user experience as presented by Lene Kristin 
Thuseth on 2015-01-16 would however be more than sufficient. It is imperative 
that mytimelines.org is able to handle larger datasets.

\begin{figure}[H]
  \centering
  \begin{scale}{0.3}
    \input{fig/overload.pdf_t}
  \end{scale}
  \caption{A timeline without filtering results in an information overload.}
  \label{fig:overload}
\end{figure}

Figure~\ref{fig:overload} demonstrates how a timeline of all of DBpedia might 
look like. The programmer art is admittedly a slight degradation of the user 
interface presented in January.

\begin{figure}[H]
  \centering
  \begin{scale}{0.3}
    \input{fig/filtered.pdf_t}
  \end{scale}
  \caption{A filtered timeline.}
  \label{fig:filtered}
\end{figure}

\begin{figure}[H]
  \centering
  \begin{scale}{0.5}
    \input{fig/filters.pdf_t}
  \end{scale}
  \caption{Creating a new filter.}
  \label{fig:filters}
\end{figure}

All of DBpedia will be initially messy and cumbersome to use without the 
option to filter on categories. This was not present in the January 
presentation, so we have taken the liberty of augmenting the design in 
Figure~\ref{fig:filtered}. This allows fine-grained control of what data is 
presently presented.

\begin{figure}[H]
  \centering
  \begin{scale}{0.3}
    \input{fig/focussed.pdf_t}
  \end{scale}
  \caption{A focussed timespan.}
  \label{fig:focussed}
\end{figure}

Finally, Figure~\ref{fig:focussed} shows an expanded timespan, wherein the 
preamble of the relevant Wikipedia article is displayed along with the 
relevant article image.
