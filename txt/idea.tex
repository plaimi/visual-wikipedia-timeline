We may scrape all of DBpedia, and record all of the entries with time metadata 
in tempuhs. This data may in turn be visualised using mytimelines.org.

In fact, DBpedia is already scraped and put into tempuhs, with metadata 
describing cataloguing the entries. The cataloguing is important for being 
able to filter the immense amounts of data.

Given a perfect universe with no computational bottlenecks, tempuhs is already 
fit for fight. Notwithstanding, due to the quantity of entries contained 
within DBpedia, a pragmatic timeline will require some improvements in tempuhs 
in order to serve it quickly enough. It needs to be partially served. It 
should furthermore in all likelihood be cached in a clever way. For this it 
might be fruitful to introduce the concept of read-only timespans to tempuhs.

As for mytimelines.org, it really is not up to par to handle this amount of 
data, or anything anywhere near this right now. timeline.js simply won't cut 
it.
